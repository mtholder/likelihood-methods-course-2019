\documentclass[11pt]{article}
\usepackage{graphicx}
\usepackage{amssymb}
%\usepackage[nomarkers]{endfloat}
\usepackage{natbib}
\usepackage{setspace}
\usepackage{wasysym}
\usepackage{wrapfig}
\pagestyle{empty}
\textwidth = 7 in
\textheight = 9 in
\oddsidemargin = -0.5 in
\evensidemargin = -0.5 in
\topmargin = 0.0 in
\headheight = 0.0 in
\headsep = 0.0 in
\parskip = 0.1in
\parindent = 0in
\renewcommand{\Pr}{\mathbb{P}}
\usepackage{paralist}
\usepackage{url}
\newcommand{\href}[2]{\url{#2}}
\usepackage{tikz}
  \usetikzlibrary{shapes.geometric}
  \usetikzlibrary{arrows.meta,arrows}
  \usetikzlibrary{positioning,automata}
\usepackage{hyperref}
\hypersetup{backref,   linkcolor=blue, citecolor=red, colorlinks=true, hyperindex=true}
\begin{document}
\section*{Homework \#7} (due Monday, May 6)

You can use the shortened \href{https://github.com/mtholder/likelihood-methods-course-2019/raw/master/homework/short_mutt_gamete_event.tsv}{short\_mutt\_gamete\_event.tsv (click on this link and download)} data for this
homework
(see hw 6 for description of the data format).


We'll look at the same model that we used in the last homework,
    but we'll take a Bayesian perspective.
So you'll need to state prior probability distributions for all
    input parameters/models.

\#1. Implement MCMC for the two-parameter ($r$ and $w$) model. Report the posterior means and 95\% credible intervals for the 2 parameters.


\#2. Report the PRSF (gelman diagnostic) for your MCMC runs.

\#3. Implement a reversible-jump MCMC version of your code to examine the $w=0$ as a special case submodel.
Report the probability that $w > 0$.
You'll need to design some move that takes you from the one parameter model
($w=0$) to the two parameter ($w>0$), and derive the Hasting's
ratio for that move.

References:
\begin{itemize}
 \item \href{http://phylo.bio.ku.edu/slides/hastingsRatio.pdf}{hastingsRatio.pdf}
\item \href{https://github.com/mtholder/likelihood-methods-course-2019/tree/master/code}{code directory for MTH's lectures}. {\tt coda.R} and
{\tt continuous-mcmc.py} are particularly relevant.
\item \url{http://patricklam.org/teaching/convergence_print.pdf}
\end{itemize}
\end{document}


