\documentclass[11pt]{article}
\usepackage{graphicx}
\usepackage{amssymb}
%\usepackage[nomarkers]{endfloat}
\usepackage{natbib}
\usepackage{setspace}
\usepackage{wasysym}
\usepackage{wrapfig}
\pagestyle{empty}
\textwidth = 7 in
\textheight = 9 in
\oddsidemargin = -0.5 in
\evensidemargin = -0.5 in
\topmargin = 0.0 in
\headheight = 0.0 in
\headsep = 0.0 in
\parskip = 0.1in
\parindent = 0in
\renewcommand{\Pr}{\mathbb{P}}
\usepackage{paralist}
\usepackage{url}
\newcommand{\href}[2]{\url{#2}}
\usepackage{hyperref}
\hypersetup{backref,   linkcolor=blue, citecolor=red, colorlinks=true, hyperindex=true}
\begin{document}
\section*{Homework \#4} (due Monday, March 18)

\#1. You want to estimate the relative frequency of the rare white color morph 
in a populations of {\em Cirsium palustre} in France; most of the flowers are purple in this
species.
So you want to estimate some parameter $p$, which is the probability that a randomly selected
individual will be a white flower.
After obtaining a generous travel award, you collect individuals in two sampling events.
The data for a total of $461$ plants are:\\
\begin{tabular}{|l|r|r|r|}
\hline
Event & sample size & \# plants with white flowers & \# plants with purple flowers \\
\hline
 Sampling event 1 & $n_1 = 199$ & $x_1=4$ & 195 \\
Sampling event 2  & $n_2 = 262$ &$x_2=8$ & 254 \\
\hline
\end{tabular}\\

(A) Write a likelihood equation $p$ for a sample generically, then given the equation with the numbers for the first sample.

(B) Consult the \url{https://en.wikipedia.org/wiki/Conjugate_prior#Table_of_conjugate_distributions} and then choose an appropriate prior distribution.
You have to pretend you didn't see the data, but just thought that the white morph was rare.
There is not one specific answer I'm looking for here -- it is {\em your} prior.

(C) If you analyzed your data after the first sample, what would your posterior distribution
on $p$ have been?

(D) What is you 95\% credible interval for $p$ after seeing the first sample? 

(E) What is the {\em maximum a posteriori} (MAP) estimate of $p$ after seeing the first sample?

(F) Imagine that you take your prior from the posterior sampling event and use it as your prior
for the second sampling event. 
Give the posterior, MAP, and 95\% C.I. after considering the second sample after the first sample.

(G) Now consider what would you would have gotten (posterior, MAP, and 95\% C.I.) if you
had started with your naive prior and analyzed all of the data at one step. 
Do you get a different posterior distribution?

\vskip 15em (Question \#2 is on second page)
\newpage
(2) You are either a botanist or someone who enjoys pretending to be a botanist.
You want to determine if there is evidence of crowding effects in reproductive success in your
favorite annual  angiosperm.
You set up an experiment in which you select a set of experimental plots which do not have 
    any seeds of the plant.
Then you transplant 1, 2, or 3 plants into each plot, and you allow them to grow for the year.
Each plot has a barrier to prevent seeds from entering from the outside.
The next year, you painstakingly count the number of small plants of your favorite species
    that emerge.

Your data:\\
\begin{tabular}{|c|c|c|}
\hline
Plot & \# plants grown last year in the plot & \# of tiny plants this year \\
\hline
1 & 1 & 19\\
2 & 2 & 25 \\
3 & 3 &  39\\
4 & 1 & 17 \\
5 & 2 & 28\\
6 & 3 & 35\\
\hline
\end{tabular}\\

For the sake of this question, let's just assume that the number of plants should follow
    a Poisson model with the expectation set by either a null model or an alternative.

You would like to test the null hypothesis ($H_0$) of no evidence of a competitive interaction.
In other words, if the null is correct the expected number of plants in this year is 
simply a linear function of the number of plants in the plot last year.
Furthermore, it is a linear function with a $y$-intercept of 0 (because we are assuming there
was no seed bank from prior years, so you'd get no plant this year if you hadn't put
any into the ground the previous year).
So the null model has some slope that is unknown but represents the number of 
    plants you expect to get in year $t+1$ from each plant that you put in the ground in year $t$.

Test that null against 2 alternatives.

The first alternative hypothesis, $H_1$, states that: (a) there is some expected reproductive success 
    for the first plant in a plot, but (b) if a plot has $>1$ progenitor planted, then the expected number of plants in
    the next year is a linear function of the number of ``extra'' parental plants (the 
    number of plants over the first plant).


The third alternative ($H_2$) is simply that you have different expected numbers of progeny
    plants based on how many plants you put in the ground in the  first year. 
So with $H_2$ have 3 expected numbers with no constrained relationship between them.
(This may be the easiest hypothesis to start working on).


Perform a maximum likelihood hypothesis test comparing these 3 models.
Report the parameter estimates, and log-likelihoods, and the test statistics
    used to decide which model is preferred.


\end{document}


