\documentclass[11pt]{article}
\usepackage{graphicx}
\usepackage{amssymb}
%\usepackage[nomarkers]{endfloat}
\usepackage{natbib}
\usepackage{setspace}
\usepackage{wasysym}
\usepackage{wrapfig}
\pagestyle{empty}
\textwidth = 7 in
\textheight = 9 in
\oddsidemargin = -0.5 in
\evensidemargin = -0.5 in
\topmargin = 0.0 in
\headheight = 0.0 in
\headsep = 0.0 in
\parskip = 0.1in
\parindent = 0in
\renewcommand{\Pr}{\mathbb{P}}
\usepackage{paralist}
\usepackage{url}
\newcommand{\href}[2]{\url{#2}}
\usepackage{tikz}
  \usetikzlibrary{shapes.geometric}
  \usetikzlibrary{arrows.meta,arrows}
  \usetikzlibrary{positioning,automata}
\usepackage{hyperref}
\hypersetup{backref,   linkcolor=blue, citecolor=red, colorlinks=true, hyperindex=true}
\begin{document}
\section*{Homework \#6} (due Wednesay, April 17)

You can use the \href{https://github.com/mtholder/likelihood-methods-course-2019/raw/master/homework/mutt_gamete_event.tsv}{mutt_gamete_event.tsv} data for this
homework.
In that tab-separated spreadsheet, I have cleaned the 
    original data into a series of events.
Each event is either a recombination event (code 'R' 
    in the first column) or an end of the chromosome
    event (code 'E').
In the second column, we have the number of bases
    that separate the event from the previous
    event.

This encoding looses the info about which stretches of the
    DNA are Dingo or Labrador.
But because that information did not affect the likelihood,
    it is easier to not deal with it.
The chromosome identity is also lost.


Email your code in with the homework.
You can write your own code or use the 
    \href{https://github.com/mtholder/likelihood-methods-course-2019/blob/master/homework/hw6-template.py}{python template}
or the 
    \href{https://github.com/mtholder/likelihood-methods-course-2019/blob/master/homework/hw6-template.R}{R template} provided.

\#1. Find the MLE of the per-base recombination probability, $r$, using
numerical methods.
    (Ideally this number will agree with what you calculated analytically
    in the previous homework).


\#2. Use the 1.92 $\ln L$ drop rule to find a 95\% confidence interval
    for $r$.

\#3. Consider the following extension:
Interference is the genetics name for ``non-independence of recombination
    events.''
A very simple model of interference is that 
    $r$ describes the per-base recombination probability for ``uninhibited''
    DNA, but a recombination event inhibits another recombination event
    from happening within a window around the point of recombination.
Let's use $w$ to denote the length of the window of inhibition.
This zone of inhibition acts before and after each recombination event.

For example, if $w=2$, you could imagine a state space like this:\\
\begin{tikzpicture}
[->,>=stealth',shorten >=1pt,auto,
 node,semithick, 
 every state/.style={fill=red,draw=none,text=black,shape=ellipse}]
\node[state, fill=orange] (A) {Unhib.~Dingo};
%\node[state, fill=orange,text=black] (B) [right=of A] {Inhib.~Dingo 2B};
%\node[state, fill=orange,text=black] (C) [right=of B] {Inhib.~Dingo 1B};
\node[state, fill=black,text=white] (D) [right=of A] {Inhib.~Labra 1A};
\node[state, fill=black,text=white] (E) [right=of D] {Inhib.~Labra 2A};
\node[state, fill=black,text=white] (F) [below=of E] {Unhib.~Labra};
\node[state, fill=orange,text=black] (G) [below=of A] {Inhib.~Dingo 2A};
\node[state, fill=orange,text=black] (H) [right=of G] {Inhib.~Dingo 1A};
%\node[state, fill=black,text=white] (I) [below=of H] {Inhib.~Labra 1B};
%\node[state, fill=black,text=white] (J) [right=of I] {Inhib.~Labra 2B};
\path (A) edge node {$r$} (D) ;
\path (A) edge [loop above] node {$1-r$} (A) ;
%\path (B) edge node {$1$} (C) ;
%\path (C) edge node {$1$} (D) ;
\path (D) edge node {$1$} (E) ;
\path (E) edge node {$1$} (F) ;
\path (F) edge [loop below] node {$1-r$} (F) ;
\path (F) edge node {$r$} (H) ;
%\path (J) edge node {$1$} (I) ;
%\path (I) edge node {$1$} (H) ;
\path (H) edge node {$1$} (G) ;
\path (G) edge node {$1$} (A) ;
\end{tikzpicture}\\
where the ``Unhib.'' states are outside of the window and can
be subject to a random recombination trigger event.
Note that the only real randomness is in the stretches of
    ``Uninhibited'' bases.
The transitions in the inhibited states all occur with probability 1.

The start state for the chromosome would be in either 
    the ``Unhibited Dingo'' or
    ``Uninhibited Labrador'' state with probability 0.5 

{\bf The task will be to find the MLE of $w$ and $r$, and do a LRT
    to see if this model of interference fits better than
    the model with no interference effects.
}

Fortunately, we don't have to draw a complicated state-space 
    diagram for every different value of $w$.
You just need to extend your likelihood model such that:
\begin{compactenum}
    \item If 2 recombination events happen within $w$ of each
        other, the likelihood is 0; and
    \item The probability of no recombination at between
        any bases that are within $w$ of a recombination event
        is 1.
\end{compactenum}

This is a 2-parameter optimization problem ($r$ and $w$ must
    be jointly optimized).
Also note that $w$ is an integer, but numerical optimizers 
    will send in floating point numbers for $w$.
I would recommend just rounding the $w$ parameter 
    down to an integer in the likelihood.
This will mean that the likelihood function is a step function
    in the $w$ dimension.
This can create optimization problems.

One alternative is to get a rough estimate of $w$ using the 2-dimensional
    numerical optimizer, and then use for-loops to examine
    higher and lower values of $w$ around the initial
    rough guess.

Another possibility helpful hint: you might be able
    to figure out the largest possible $w$ for your data
    set and provide that as an initial bracketing constraint.


\end{document}


